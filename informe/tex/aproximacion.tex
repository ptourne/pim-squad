\section{Greedy}

Proponemos dos aproximaciones extra. La primera es un algoritmo greedy que calcula la cantidad de los $m$ subconjuntos $B$ en la que aparece cada jugador y luego agrega el jugador con más apariciones de cada subconjunto. La primera operación tiene complejidad $O(k \times m)$, con $k$ el promedio de jugadores por subconjunto ($k \leq n$). Con grupos más chicos, el algoritmo tiende a lineal ($O(m)$) y con grupos más grandes, a $O(m\times n)$. La segunda operación tiene la misma complejidad. Entonces, la complejidad total es $O(k \times m)$ y resulta pseudopolinomial. % O polinomial?

Esta resulta ser una $m$-aproximación, ya que este algoritmo puede elegir $m$ jugadores, cuando la solución óptima conta con uno solo. Esto se presenta cuando se tienen $m$ grupos que comparen $m$ jugadores ($\left|B_1 \cap B_2 \dots \cap B_m\right| = m$). Todos los jugadores dentro de la intersección pertenencen a $m$ grupos y el algoritmo propuesto elige, para cada grupo, el jugador que pertenece a más grupos. Como cada grupo tiene $m$ jugadores con el mismo puntaje, podría elegir a cualquiera, y si para cada grupo elige uno distinto, terminaríamos con un resultado de tamaño $m$, mientras que bastaba con elegir uno solo.

La segunda aproximación comienza de la misma manera, calculando la cantidad de apariciones de cada jugador en cada subconjunto. Luego agrega el jugador con más apariciones entre todos y quita las apriciones de los subconjuntos que ya cubre al resto de los jugadores. Realiza esta operación hasta quedarse sin jugadores o que el jugador encontrado tenga cero apariciones restantes. Ya vimos quie la complejidad de la primera operación es $O(k \times m)$. La segunda tiene complejidad $O(j \times g \times m)$, con $j$ la cantidad de jugadores de la solución, $j \leq m$, y $g$ la cantidad promedio de grupos que cubre cada jugador $g \leq k$. Entonces, la complejidad total es $O(k \times m + j \times g \times m)$. %?

    