\section{Conclusiones}

Finalmente, consideramos que la solución más adecuada para abordar la problemática presentada por Scaloni sería a través del empleo de técnicas como Backtracking o Programación Lineal Entera, siempre y cuando el número de medios no sea excesivamente alto. 

No obstante, en situaciones que involucren conjuntos de medios considerablemente extensos, las aproximaciones obtenidas a través de algoritmos como los métodos greedy o la Programación Lineal pueden ofrecer una alternativa viable para obtener resultados en tiempos reducidos. Sin embargo, se debe tener en cuenta que, si bien estas soluciones pueden ser efectivas en términos de eficiencia computacional, pueden distanciarse significativamente de la solución óptima. Este fenómeno es particularmente notable en el caso de la Programación Lineal, la cual representa una (b-1)-Aproximación (siendo b el número máximo de elementos en los conjuntos de medios), siendo una diferencia significativa respecto a la solución óptima.
