\section{Introducción}

El propósito de este informe radica en llevar a cabo un análisis de los 
distintos enfoques y diseños aplicados para abordar la problemática presentada por 
Scaloni. Este desafío se centra en la determinación del conjunto mínimo de jugadores,
denotado como $C$, requerido para satisfacer las demandas de cada medio, asegurando al 
menos la presencia de un jugador favorito por cada uno. Es importante destacar que este 
problema se enmarca como un caso específico del 'Hitting Set Problem'. 
En consecuencia, a lo largo de este informe, se realizará una análisis de múltiples soluciones con el fin de resolver este último. 

\subsection{Información de la problemática}

El enunciado propuesto, nos da la siguiente información:

\begin{itemize}

    \item Scaloni tiene a su disposición el conjunto $A$ de $n=43$ jugadores $a_1, a_2, \dots, a_{43}$.
    \item Existen $m$ medios, cada uno con un grupo de jugadores favorito $B_1, B_2, \dots, B_m$, 
    $(B_i \in A \forall i)$
    \item Se quiere el subconjunto $C \in A$ de menor tamaño tal que C tenga al menos un elemento de cada $B_i$ (es decir, $C \cap B_i$) 
    \item Scaloni necesita obtener el grupo $C$ más pequeño de jugadores de tal forma que cada medio $B_i$ tenga al menos un jugador en el equipo.

\end{itemize}
