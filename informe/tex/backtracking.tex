\section{Solución por backtracking}

Para poder plantear el álgoritmo de backtracing procedimos a seguir el esquema proporcionado por la cátedra:
\begin{enumerate}
    \item Pruebo si la solucion parcial es solución y es mejor que la actual entontrada:
    \begin {enumerate}
        \item Si lo es, actualizo la solución actual y la devuelvo
        \item Si no lo es, avanzo si puedo
    \end{enumerate}
    \item Pruebo si la solución parcial es válida
    \begin {enumerate}
        \item Si no lo es, retrocedo y vuelvo a 3)
        \item Si lo es, llamo recursivamente y vuelvo a 1)
    \end{enumerate}
    \item Prosigo explorando la solución parcial llamando recursivamente
\end{enumerate}

Para la solución por backtracking consideramos relevantes las siguientes variables:
\begin{itemize}
    \item Los subconjuntos B_i
    \item Solución parcial
    \item La solución actual óptima 
    \item El índice (i_b) del subconjunto B_i en cuestión. 
\end{itemize}

En cada instancia recursiva del algoritmo, se inicia con la evaluación de la compatibilidad de la solución parcial con el problema de Hitting-Set. En caso de ser compatible y si su cardinalidad es inferior a la de la solución actual, esta solución parcial se considera la nueva solución óptima (actual).

Por otro lado, si la solución parcial no es compatible con el problema planteado, se procede a descartarla únicamente si su cardinalidad es mayor o igual que el de la solución actual. En este escenario, se implementa la técnica de backtracking, regresando al llamado anterior y desechando la exploración de esta rama de solución.

En el caso específico en el que la solución parcial no es compatible pero aún puede representar una solución óptima potencial, se lleva a cabo una verificación adicional. Se examina el subconjunto B_i (donde i=i_b) para garantizar la inclusión de al menos un jugador en la solución final. Si este requisito no se cumple, se procede a evaluar individualmente la inclusión de cada jugador presente en B_i para determinar cuál de ellos conduce a la obtención de la solución óptima.