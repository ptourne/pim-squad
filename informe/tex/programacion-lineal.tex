\section{Solución por Programación Lineal}

\subsection{Programación Lineal Entera}

También se puede reducir Hitting-Set Problem a Programación Lineal Entera de la siguiente manera:

\begin{enumerate}

    \item Para cada elemento $e_i$ del universo $U=\left\{e_1,e_2,\dots,e_{n}\right\}$, se crea una variable $y_{i}$ que puede tomar los valores $0$ o $1$.
    
    \item Se define una función \(f_{j} = \sum_{i\mid e_i \in S_j} y_i\) para cada conjunto \(S_{j}\), y se establece una restricción que acota a \(f_{j}\) en el rango \(1 \leq f_{j} \leq |S_{j}|\).

    \item  El objetivo radica en minimizar la función $f=\sum_{i=0}^{n}y_{i}$.

\end{enumerate}

Luego, el conjunto solución $C$ será

\[
    C = \{e_i \in U \mid y_{i} = 1\}
\]

La complejidad de esta reducción es $O(m \times b)$, con $b$ el cardinal del conjunto más grande, ya que se debe recorrer todos los subconjuntos $B_i$ y, por cada uno, recorrer cada elemento. Este proceso es polinomial, y la complejidad de Programación Lineal Entera es exponencial.

\[\text{Hitting-Set} \leq _{p} \text{Programación Lineal Entera}\]

El codigo de la reducción es el siguiente:

\lstinputlisting[language=Python, firstline=4, lastline=32]{../algoritmos/programacion_lineal_entera.py}

\subsection{Aproximación por Programación Lineal Continua}

Otra estrategia para abordar la solución del Hitting-Set Problem implica el enfoque de la Programación Lineal continua. Esta aproximación se asemeja a la reducción previamente descrita, con la distinción fundamental de que las variables 
$y_i$ pueden asumir valores reales dentro del intervalo $0$ a $1$.

Posteriormente, para toda variable $y_i$ que en la solución del problema de Programación Lineal tengan un valor que exceda el umbral del $1/b$, siendo $b$ el cardinal máximo de todos los conjuntos $S_j$, se incluye $e_i$ a la solución final $C$ del Hitting-Set Problem.

Luego, el conjunto solución $C$ será

\[
    C = \left\{e_i \in U \mid y_{i} \geq \frac{1}{b}\right\}
\]

Por la mismas consideraciones aplicadas en la reducción de la Programación Lineal Entera, la complejidad asociada a la reducción de Hitting Set a Programación Lineal Continua es de $O(m\times b)$.
Sin embargo, la complejidad de Programación Lineal Continua es $O(n^9)$ lo cual es polinomial, a diferencia de la Entera. 

El codigo de la reducción es el siguiente:

\lstinputlisting[language=Python, firstline=3, lastline=38]{../algoritmos/programacion_lineal_continua.py}

Con el objetivo de evaluar qué tan buena aproximación es la Programación Lineal Continua respecto a la Programación Lineal Entera, se propuso el siguiente escenario que resalta las diferencias entre las soluciones obtenidas por ambos algorítmos:

Consideremos dos conjuntos que representan medios diferentes, denominados $S_1$ y $S_2$. Estos conjuntos se componen de la siguiente manera:
\begin{enumerate}
    \item $S_1$ con tamaño $b$, expresado como $S_1=\left\{x_1, x_2, ..., x_{b-1},  x_b\right\}$
    \item $S_2$ con tamaño $b-1$, representado como $S_2=\left\{x_1, x_2, ..., x_{b-1}\right\}$
\end{enumerate}

En el contexto de Programación Lineal Continua (PLC), una solución posible puede ser la asignación $x_1 = x_2 = ... = x_{b-1} = 1/(b-1)$ y $x_b=0$.

En esta situación, dado que $\frac{1}{b} < \frac{1}{b-1}$, las variables del $S_2$, al ser redondeadas, resultarían en 1, siendo todas incluidas en la solución final.

Por otro lado, la solución proporcionada por la Programación Lineal Entera (PLE), que representa la solución óptima, presentaría, por ejemplo, la siguiente asignación de valores:  $x_1 = 1$ y  $x_2 = ... = x_{b} = 0$. 

No obstante, al obtenerse una aproximación mediante PLC que establece $x_1 = x_2 = ... = x_{b-1} = 1$ se deriva la siguiente relación de aproximación: $r(A)=\frac{A(I)}{z(I)}=\frac{b-1}{1}$

En consecuencia, $b-1$, se establece como una cota superior de la aproximación. 

