\section{Programación Lineal}

\subsection{Programación Lineal Entera}

También se puede reducir Hitting-Set Problem a Programación Lineal de la siguiente manera:

\begin{enumerate}

    \item Para cada elemento $i$ del universo $U$, se crea una variable $y_{i}$ que puede tomar los valores $0$ o $1$.
    
    \item Se define una función \(f_{j} = \sum_{e \in S_j} e\) para cada conjunto \(S_{j}\), y se establece una restricción que acota a \(f_{j}\) en el rango \(1 \leq f_{j} \leq |S_{j}|\).


    \item  El objetivo radica en minimizar la función $f=\sum_{i=0}^{n}y_{i}$.

\end{enumerate}

Luego, el conjunto solución $C$ será

\[
    C = \{e_i \in U \mid y_{i} = 1\}
\]

La complejidad de esta reducción es $O(N \times U)$ ya que se debe recorrer todos los subconjuntos $B_i$ y, por cada uno, recorrer cada elemento. Este proceso es polinomial, y la complejidad de Programación Lineal Entera es exponencial.

\subsection{Programación Lineal Continua}

Otra estrategia para abordar la solución del Hitting-Set Problem implica el enfoque de la Programación Lineal continua. Esta aproximación se asemeja a la reducción previamente descrita, con la distinción fundamental de que las variables 
$y_i$ pueden asumir valores reales dentro del intervalo $0$ a $1$.

Cada variable $y_i$, en la solución del problema, representa una probabilidad de que el jugador correspondiente sea parte de la solución óptima. Posteriormente, se lleva a cabo una consideración respecto a estos valores, donde aquellos que excedan el umbral del $0.5$ son incluidos como componentes de la solución óptima.

Luego, el conjunto solución $C$ será

\[
    C = \left\{e_i \in U \mid y_{i} \geq \frac{1}{2}\right\}
\]

Por la mismas consideraciones aplicadas en la reducción de la Programación Lineal Entera, la complejidad asociada a la reducción de la Programación Lineal Continua al Hitting Set es de O(N×U), representativa de un tiempo de ejecución polinomial.
Sin embargo, la complejidad de Programación Lineal Continua es $O(n^9)$ lo cual es polinomial, a diferencia de la Entera. 


