\documentclass{estilo}
\usepackage[spanish]{babel}
\usepackage{graphicx}
\usepackage{float}
\usepackage{amsmath}        % para los vectores columnas
\usepackage{amsfonts}       % para las negrita de pizarra
\usepackage{amssymb}        % para simbolos matematicos
\usepackage{hyperref}       % para utilizar referencias
\usepackage{multirow}       % para las tablas
\usepackage{dsfont}
\usepackage{listings}
\usepackage{xcolor}
\usepackage{hyperref}

\usepackage{tocloft}

\definecolor{codegreen}{rgb}{0,0.6,0}
\definecolor{codegray}{rgb}{0.5,0.5,0.5}
\definecolor{codepurple}{rgb}{0.58,0,0.82}
\definecolor{backcolour}{rgb}{0.95,0.95,0.92}
\lstdefinestyle{mystyle}{
    backgroundcolor=\color{backcolour},   
    commentstyle=\color{codegreen},
    keywordstyle=\color{magenta},
    numberstyle=\tiny\color{codegray},
    stringstyle=\color{codepurple},
    basicstyle=\ttfamily\footnotesize,
    breakatwhitespace=false,         
    breaklines=true,                 
    captionpos=b,                    
    keepspaces=true,                 
    numbers=left,                    
    numbersep=5pt,                  
    showspaces=false,                
    showstringspaces=false,
    showtabs=false,                  
    tabsize=2
}
\lstset{style=mystyle}

\usepackage{enumitem,multicol,setspace}
\newcounter{subenum}[enumi] % para las multicolumnas
\renewcommand{\thesubenum}{\arabic{subenum}}
\usepackage[nomessages]{fp}
\FPeval\thecolwidth{round(1/4:4)}% Specify number of columns -> column width
\newcommand{\newitem}[1]{%
  \refstepcounter{subenum}%
  \parbox{\dimexpr\thecolwidth\linewidth-.5\columnsep}{%
    \makebox[\labelwidth][r]{(\thesubenum)\hspace*{\labelsep}}%
    #1}\hfill%
}

\usepackage{scalerel,stackengine} % para el sombrero
\stackMath
\newcommand\rhat[1]{%
\savestack{\tmpbox}{\stretchto{%
  \scaleto{%
    \scalerel*[\widthof{\ensuremath{#1}}]{\kern-.6pt\bigwedge\kern-.6pt}%
    {\rule[-\textheight/2]{1ex}{\textheight}}%WIDTH-LIMITED BIG WEDGE
  }{\textheight}% 
}{0.5ex}}%
\stackon[1pt]{#1}{\tmpbox}%
}
\parskip 1ex

\usepackage{mathtools}      % floor y ceil
\DeclarePairedDelimiter\ceil{\lceil}{\rceil}
\DeclarePairedDelimiter\floor{\lfloor}{\rfloor} 

\usepackage[style=authoryear-comp]{biblatex}


\begin{document}

\maketitle

\newpage

\tableofcontents

\newpage

\section{Introducción}

En el presente informe, se tiene como objetivo llevar a cabo un análisis del diseño 
implementado para abordar la problemática planteada por Scaloni, que consiste en la 
optimización máxima del tiempo total empleado en la realización de un análisis exhaustivo 
de sus rivales. A lo largo de este, se expondrán los criterios empleados y se proporcionará 
su justificación.


\subsection{Información de la problemática}

El enunciado propuesto, nos da la siguiente información:

\begin{itemize}

    \item Cada compilado lo debe analizar Scaloni y alguno de sus ayudantes.

    \item El análisis del rival $i$ le toma $s_i$ tiempo a Scaloni y luego $a_i$ tiempo al ayudante.
    
    \item Al momento en que Scaloni haya terminado de analizar el $i$ ésimo compilado, comenzará
    inmediatamente algún ayudante a analizarlo, para no desperdiciar ningún segundo.

    \item Scaloni cuenta con $n$ ayudantes, siendo $n$ la cantidad de rivales a analizar. Además, cada
    ayudante puede ver los compilados completamente en paralelo a Scaloni y a los respectivos ayudantes.

    \item Sólo un ayudante verá el compilado, dado que no aporta mayor ganancia que dos ayudantes lo vean.

\end{itemize}

\section{Soluciones propuestas}

Para realizar de forma óptima el trabajo propuesto, consideramos dos cuestiones:

\begin{itemize}

    \item El orden en que Scaloni visualiza los compilados no incide en el tiempo que le toma a él en
    finalizar la revisión de los mismos. Tomando esto en cuenta, el ordenamiento que empleamos ignora el
    tiempo que le tarda al DT ver cada compilado. 

    \item Los asistentes realizan el análisis de cada uno de los compilados asignados en paralelo. Por lo 
tanto, el tiempo que se invierte en la revisión de un compilado específico de máxima duración, 
puede ser aprovechado de manera tal que este sea visto mientras Scaloni se dedica a la revisión
de otros compilados. De esta forma, nos aseguramos que se minimice el tiempo que suman los ayudantes 
en la resvisión total. 


\end{itemize}

\textbf{Para ello propusimos el siguiente algoritmo:}

En primer lugar, hemos definido una clase llamada \texttt{Compilado} para modelar el compilado de cada 
oponente, con los atributos \texttt{tiempo\_scaloni} y \texttt{tiempo\_ayudante}, que almacenan 
el tiempo que le lleva analizarlo a Scaloni y a algún ayudante, respectivamente.

\begin{lstlisting}[language=Python]
class Compilado:
    def __init__(self, scaloni, ayudante):
        self.tiempo_scaloni = scaloni
        self.tiempo_ayudante = ayudante
\end{lstlisting}

De esta forma, y teniendo en cuenta los criterios previamente detallados, definimos la función
\texttt{compilados\_ordenados\_de\_forma\_optima} que recibe como parámetro un arreglo con
elementos de la clase \texttt{Compilado}. Esta ordena el arreglo en función del tiempo requerido
por los asistentes para visualizar cada compilado, en orden descendente. 

\begin{lstlisting}[language=Python]
def compilados_ordenados_de_forma_optima(compilados):
    return sorted(compilados, key=lambda compilado: compilado.tiempo_ayudante, reverse=True)
\end{lstlisting}

\subsection{Cota de complejidad temporal}

El algoritmo presentado tiene una complejidad temporal de $O(n log(n))$. Esto se debe a 
que seguimos los siguientes pasos:
Ordenamos los compilados según el tiempo de análisis de los ayudantes. Para ello utilizamos
\href{https://docs.python.org/3/howto/sorting.html}{sorted} de la librería de python, que usa 
por detrás el algoritmo de
\href{https://en.wikipedia.org/wiki/Timsort}{Timsort}, teniendo este una complejidad de $O(n log(n))$.

\section{Mediciones}

Se realizaron mediciones en base a crear compilados aleatorios que llevan analizarlos entre 1 a 
99999 a Scaloni y a sus ayudantes, yendo de 10 en 10 hasta 1000 elementos.

\begin{figure}[H]
    \centering
    \includegraphics[width=1\textwidth]{img/tiempos\_ejecucion.png}
\end{figure}

Los comparamos con los gráficos de una función lineal y una lineal logarítmica, sabiendo que 
se acercaría más a una lineal\_logarítmica por la complejidad de nuestra función.
\section{Conclusiones}

Finalmente, consideramos que la solución óptima para abordar la problemática de Scaloni sería que
éste analizara los compilados en función del tiempo requerido por los asistentes para vi\-sualizar
cada uno, organizándolos en orden descendente.
Esta estrategia permitiría resolver el problema con una complejidad algorítmica de orden 
$\operatorname{O}(n\log{n})$.
Este enfoque garantiza la máxima eficiencia en la visualización de los compilados y permite que el 
tiempo invertido por Scaloni y sus asistentes se administre de manera óptima.


\newpage
\end{document}