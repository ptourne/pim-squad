\section{Mediciones}

Para las mediciones, creamos listas de compilados de ejemplo de forma aleatoria. Decidimos que el tiempo
de análisis de un compilado fuera un valor aleatorio entre 1 y 99'999, basándonos en los datos de ejemplo
provistos por la cátedra
Medimos el tiempo de ejecución de nuestro algoritmo para listas de 10 hasta 10'000 elementos
y luego los graficamos.

\begin{figure}[H]
    \centering
    \includegraphics[width=1\textwidth]{img/tiempos\_ejecucion.png}
\end{figure}

Calculamos una regresión lineal y una regresión lineal logaritmica y graficamos sus curvas sobre los datos.
De esta forma pudimos comprobar empíricamente que la complejidad tiende a $\operatorname{O}(n\log{n})$.