\section{Mediciones}

Para las mediciones, creamos listas de compilados de ejemplo de forma aleatoria. Decidimos que el tiempo
de análisis de un compilado fuera un valor aleatorio entre 1 y 99'999, basándonos en los datos de ejemplo
provistos por la cátedra.
Medimos el tiempo de ejecución de nuestro algoritmo para ciertas cantidades de compilados.

Calculamos una regresión lineal y una regresión lineal logaritmica que se ajuste a nuestros datos y graficamos sus curvas.
Ahora, debemos comprobar cuál de las dos curvas se ajusta mejor a nuestros datos. Para ello, usamos la raíz del
error cuadrático medio, que nos da una idea de cuán cerca están los valores predichos de los valores reales.

Al calcular el error cuadrático medio para ambas curvas, observamos que el error era prácticamente similar.
En consecuencia, decidimos llevar a cabo un análisis más detallado,
enfocándonos en diversos intervalos del gráfico con el objetivo de examinar el comportamiento de los datos en mayor
profundidad. Esta investigación nos permitió concluir que, especialmente en el caso de muestras caracterizadas
por valores reducidos, la curva lineal logarítmica se ajusta mejor a los datos observados.

Sin embargo, al considerar valores más elevados, se observó que las diferencias en los tiempos tendieron a adquirir una naturaleza lineal. 
De la misma manera, la función logarítmica se asemeja a una lineal para valores grandes, por lo que el error cuadrático medio tendía a ser el mismo
para ambas funciones. Dado este análisis, decidimos incrementar el número de muestras para cantidades de compilados menores, 
y disminuir el número de muestras para cantidades de compilados mayores. 

De esta manera, mediante el gráfico y la validación a través del error cuadrático medio, hemos comprobado que la curva lineal logarítmica 
se adapta de manera más precisa a nuestros datos.


\begin{figure}[H]
    \centering
    \includegraphics[width=1\textwidth]{img/tiempos\_valores\_bajos.png}
\end{figure}

\begin{figure}[H]
    \centering
    \includegraphics[width=1\textwidth]{img/tiempos\_valores\_altos.png}
\end{figure}

De esta forma pudimos comprobar empíricamente que la complejidad tiende a $\operatorname{O}(n\log{n})$.