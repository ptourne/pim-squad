\section{Introducción}

En el presente informe, se tiene como objetivo llevar a cabo un análisis del diseño 
implementado para abordar la problemática planteada por Scaloni, que consiste en la 
optimización máxima del tiempo total empleado en la realización de un análisis exhaustivo 
de sus rivales. A lo largo de este, se expondrán los criterios empleados y se proporcionará 
su justificación.


\subsection{Información de la problemática}

El enunciado propuesto, nos da la siguiente información:

\begin{itemize}

    \item Cada compilado lo debe analizar Scaloni y alguno de sus ayudantes.

    \item El análisis del rival $i$ le toma $s_i$ tiempo a Scaloni y luego $a_i$ tiempo al ayudante.

    \item Scaloni cuenta con $n$ ayudantes, siendo $n$ la cantidad de rivales a analizar. Además, cada
        ayudante puede ver los videos completamente en paralelo a Scaloni y a los respectivos ayudantes.

    \item Al momento en que Scaloni haya terminado de analizar el $i$-ésimo video, comenzará
        inmediatamente algún ayudante a analizarlo, para no desperdiciar ningún segundo.

    \item Sólo un ayudante verá el video, dado que no aporta mayor ganancia que dos ayudantes lo vean.

\end{itemize}
