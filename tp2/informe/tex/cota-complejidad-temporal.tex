\subsection{Cota de complejidad temporal}

El algoritmo presentado tiene una complejidad temporal de $\operatorname{O}(n^2)$.
Esto se debe a que, para obtener la solución óptima, requerimos analizar cada día de entrenamiento, dependiendo de la energía disponible
para ese día y sus variantes.\
Técnicamente lo que hacemos es recorrer la mitad de una matriz de $n*n$, siendo $n$ la cantidad de días a entrenar (igual cantidad que
los valores que toman las energías).\
Ahora, para cada día, calculamos el resultado parcial respecto a los días anteriores, es decir, observamos el máximo 