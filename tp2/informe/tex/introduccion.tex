\section{Introducción}

En el presente informe, se tiene como objetivo llevar a cabo un análisis del diseño 
implementado para abordar la problemática planteada por Scaloni, que consiste en 
definir los días en que los jugadores deban entrenarse y los días que les convenga descansar
de tal forma de tener la mayor ganancia posible. A lo largo de este, se expondrán los criterios
empleados y se proporcionará su justificación.


\subsection{Información de la problemática}

El enunciado propuesto, nos da la siguiente información:

\begin{itemize}
    \item Scaloni definió un cronograma de entrenamiento.
    \item El entrenamiento del día $i$ demanda una cantidad de esfuerzo $e_i$.
    \item El entrenamiento que corresponde al día $i$ es inamovible.
    \item  La cantidad de energía disponible para cada día va disminuyendo a medida que pasan
    los entrenamientos.
    \item La distribución de energía en cada día sigue la siguiente secuencia:  $s_1 \geq s_2 \geq ... \geq s_n$. De esta manera
    $s_i$ es la energía disponible para el $i$-ésimo día de entrenamiento consecutivo.
    \item El entrenador tiene la opción de otorgarles un día de descanso, lo que resulta en la renovación de la 
    energía de los jugadores (es decir, el próximo entrenamiento comenzaría nuevamente con la energía $s_1$, seguida
    de $s_2$, y así sucesivamente). 
    \item Si se opta por el descanso, el entrenamiento programado para ese día no se lleva a cabo, y en consecuencia, 
    no se obtiene ninguna ganancia.
    \item Si el nivel de esfuerzo $e_i$ excede la energía disponible $s_i$ en un día determinado,
     la ganancia resultante del entrenamiento es igual a la energía disponible. Caso contrario, la ganancia del entrenamiento se 
     define por el nivel de esfuerzo realizado.
    \item Dada la secuencia de energía disponible desde el último descanso $s_1 , s_2 ,..., s_n$ y el esfuerzo/ganancia de cada día 
     $e_i$, se pide determinar la máxima cantidad de ganancia que se puede obtener de los entrenamientos, considerando posibles descansos.
\end{itemize}
